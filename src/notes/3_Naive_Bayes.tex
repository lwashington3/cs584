%! Author = Len Washington III
%! Date = 9/04/24

% Preamble
\documentclass[
	number={3},
	title={Na\:ive Bayes Learning}
]{cs584notes}
\usepackage{mdsymbol}

% Document
\begin{document}

\section{Direct Learning}\label{sec:direct-learning}
\begin{itemize}
	\item Consider a \emph{distribution} \data{$D$}
	\item \data{$X$} - \emph{Instance} space, \data{$Y$} - Set of \emph{labels}. (e.g. \data{$\pm1$})
	\item Given a \emph{sample} \data{$\{ (x,y) \}_{1}^{n}$} and a \emph{loss function} \data{$L(x,y)$}, find a \emph{hypothesis}
\end{itemize}

\section{Probabilistic Model}\label{sec:probabilistic-model}
Paradigm:
\begin{itemize}
	\item \emph{Learn} a \emph{probability distribution} of the \emph{dataset}.
	\item Use it to \emph{estimate} which outcome is more likely.
\end{itemize}

Instead of learning \data{$h: X\rightarrow Y$}, \emph{learn} \data{$P(Y|X)$}.

\begin{itemize}
	\item Estimate probability from data
	\begin{itemize}
		\item Maximum Likelihood Estimate (MLE)
		\item Maximum Aposteriori Estimation (MAP)
	\end{itemize}
\end{itemize}

\section{Probability Recap}\label{sec:probability-recap}
\[ 0 \leq P(A) \leq 1 \]
\[ P(true) = 1, P(false) = 0 \]
\[ P(A \lor B) = P(A) + P(B) + P(A \land B) \]
\[ P(A|B) = \frac{P(A\land B)}{{P(B)}} \]

\section{Joint Distribution}\label{sec:joint-distribution}
Making a \emph{joint distribution} of \data{$d$} \emph{variables}
\begin{itemize}
	\item Make a \emph{truth table} listing all combinations of values of your variables (if there are \data{$d$} \emph{boolean variables} then the table will have \data{$2^{d}$ rows})
	\item For \emph{each combination} of values, say how \emph{probable} it is.
	\item The \emph{probability} must \emph{sum} up to \data{$1$}.
\end{itemize}

Once we have the Joint Distribution, we find probability of any logical expression involving these variables.

\[ \data{P(E) = \sum_{rows\ matching\ E}P(row)} \]

\section{Independence}\label{sec:independence}
When two \emph{events} do \emph{not affect} each other's \emph{probabilities}, they are called \emph{independent events}

\[ \data{ A \upvDash B \leftrightarrow P(A\land B) = P(A)\times P(B) } \]

The \emph{conditional independence} of events \data{$A$} and \data{$B$}, given \data{$C$} is:
\[ \data{ A \upvDash B | c \leftrightarrow P(A|B,C) = \frac{P(A\land B|C)}{P(B|C)} = \frac{P(A|C)\times P(B|C)}{P(B|C)} = P(A|C) } \]

\section{Bayes' Rule}\label{sec:bayes'-rule}
\begin{equation}
	\data{P(A|B) = \frac{P(B|A)\times P(A)}{P(B)}}
	\label{eq:bayes-rule}
\end{equation}
where \data{$A$} and \data{$B$} are \emph{events} and \data{$P(B)\neq0$}.
Applying \emph{Bayes' rule} for \emph{machine learning} --

\begin{equation}
	\data{P(hypothesis\ |\ evidence) = \frac{P(evidence\ |\ hypothesis)\times P(hypothesis)}{P(evidence)}}
	\label{eq:bayes-rule-ml}
\end{equation}

\section{Bayesian Learning}\label{sec:bayesian-learning}
\begin{itemize}
	\item Goal: find the \emph{best hypothesis} from some space \data{$H$} of \emph{hypotheses}, given the observed data (\emph{evidence}) \data{$D$}.
	\item Define the \emph{most probable hypothesis} in \data{$H$} to be the \emph{best}.
	\item In order to do that, we need to \emph{assume} a \emph{probability distribution} over the \emph{class} \data{$H$}.
	\item In addition, we need to know something about the \emph{relation} $\dots$
\end{itemize}

\begin{description}[font=\color{red}]
	\item[$P(h)$] -- \emph{Prior Probability} of the \emph{hypothesis} \data{$h$}. Reflects the background knowledge, before data is observed.
	\item[$P(D)$] -- \emph{Probability} that this sample of the \emph{data} is \emph{observed}.
	\item[$P(D|h)$] -- {Probability} of \emph{observing} the \emph{sample} \data{$D$}, given that \emph{hypothesis} $h$ is the \emph{target}, also referred to as \emph{likelihood}.
	\item[$P(h|D)$] -- \emph{Posterior probability} of \data{$h$}. The \emph{probability} that \data{$h$} is the \emph{target}, given that \data{$D$} has been \emph{observed}.
\end{description}

\begin{itemize}
	\item \data{$P(h|D)$} \emph{increases} with \data{$P(h)$} and \data{$P(D|h)$}.
	\item \data{$P(h|D)$} \emph{decreases} with \data{$P(D)$}.
\end{itemize}

\section{Maximum APosteriori Estimate}\label{sec:maximum-aposteriori-estimate}
\[ \data{ P(h|D) = \frac{P(D|h)\times P(h)}{P(D)} } \]
\begin{itemize}
	\item The \emph{learner} considers a \emph{set of candidate hypotheses} \data{$H$} (models) and attempts to find the \emph{most probable} one \data{$h\in H$}, given the observed data.
	\item Such maximally probable hypothesis is called \emph{maximum a posterior estimate} (MAP). Bayes theorem is used to compute it:
	\data{\begin{equation*}
	\begin{aligned}
		h_{MAP} &= \arg\max_{h\in H}P(h|D)\\
				&= \arg\max_{h\in H}\frac{P(D|h)\times P(h)}{P(D)}\\
				&= \arg\max_{h\in H}P(D|h)\times P(h)\\%}
	\end{aligned}
	\end{equation*}}
\end{itemize}

\section{Maximum Likelihood Estimate}\label{sec:maximum-likelihood-estimate}
\begin{itemize}
	\item We may assume that \emph{a priori}, \emph{hypotheses} are \emph{equally probable.}
	\[ P(h_{i}) = P(h_{j}) \forall h_{i}, h_{j} \in H \]
	\item With that assumption, we can treat \data{$\frac{P(h)}{P(D)}$} as a constant. We get the \emph{maximum likelihood estimate} (MLE):
	\data{\begin{equation*}
	\begin{aligned}
		h_{MLE} &= \arg\max_{h\in H} \frac{P(D|h)\times P(h)}{P(D)}\\
				&= \arg\max_{h\in H} P(D|h)\times P(h)
	\end{aligned}
	\end{equation*}}
	\item Here we just \emph{look for} the \emph{hypothesis} that \emph{best explains} the \emph{data}.
\end{itemize}

\section{Bayesian Classifier}\label{sec:bayesian-classifier}
\begin{itemize}
	\item \data{$f: \vec{X}\rightarrow Y$} where, instances \data{$x\in X$} is a collection of inputs --
	\[ \data{ \vec{x} = (x_{1}, x_{2}, \dots, x_{n}) } \]
	\item Given an example, \emph{assign} it the \emph{most probable} value in \data{$Y$}.
	\begin{equation*}
	\begin{aligned}
		y_{MAP} &= \arg\max_{y_{j}\in Y} P(y_{j} | x)\\
				&= \arg\max_{y_{j}\in Y} P(y_{j} | x) \dots
	\end{aligned}
	\end{equation*}
	\item Given the training data, we have to \emph{estimate} the two terms.
	\item Estimating \data{$P(y)$} is easy, e.g., under the \emph{binomial distribution assumption}, \emph{count} the number of \emph{times} \data{$y$} appears in the training data.
	\item However, it is \emph{not feasible} to estimate \data{$P(x_{1}, x_{2}, \dots, x_{n} | y)$}
	\item In this case, we have to \emph{estimate}
\end{itemize}

\section{Na\:ive Bayes Classifier}\label{sec:naive-bayes-classifier}
\emph{Assumption}: Input feature values are independent, given the target value.
\begin{equation*}
\begin{aligned}
	P(x_{1}, x_{2}, \dots, x_{n} | y_{j}) &= P(x_{1} | x_{2}, \dots, x_{n}, x_{j}) \times P(x_{2}, \dots, x_{n} | y)\\
										  &= P(x_{1} | x_{2}, \dots, x_{n}, x_{j}) \times P(x_{2}, \dots, x_{n} | y)\\
	&=\vdots\\
	&= \data{ \Pi_{i=1}^{n} P(x_{i} | y_{j}) }
\end{aligned}
\end{equation*}

\section{Gaussian Naïve Bayes}\label{sec:gaussian-naive-bayes}
Compute the \emph{mean} and \emph{standard deviation} to estimate the \emph{likelihood}.

\begin{equation*}
	\begin{aligned}
		\mu_{1} &= E[X_{1}\ |\ Y = 1] = \frac{2 + (-1.2)  + 2.2}{3} = 1\\
		\sigma_{1}^{2} &= E\left[ (X_{1} - \mu_{1})^{2} | Y = 1 \right] = \frac{(2-1)^{2} + (-1.2 - 1)^{2} + (2.2 - 1)^{2}}{3} = 2.43\\
		P(x_{1} | Y=1 ) &= \frac{1}{\sqrt{2\pi\sigma^{2}}}e^{-\frac{(x_{1} - \mu_{1})^{2}}{2\sigma^{2}}} = \frac{1}{3.91}e^{-\frac{(x_{1} - 1)^{2}}{4.86}}
	\end{aligned}
\end{equation*}

\section{Bayesian Belief Network}\label{sec:bayesian-belief-network}
\begin{itemize}
	\item Naïve Bayes classifier works with the \emph{assumption} that the values of the \emph{input features} are \emph{conditionally independent} given the \emph{target value}.
	\item This \emph{assumption} dramatically \emph{reduces} the \emph{complexity} of \emph{learning} the \emph{target function}.
	\item \emph{Bayesian Belief Network} describes the probability distribution governing a set of variables by specifying a \emph{set of conditional independence assumptions} along with a set of conditional probabilities. \emph{Conditional independence} assumptions here apply to \emph{subsets} of the \emph{variables}.
	\[ \data{ P(x_{1}, x_{2}, \dots, x_{l}\ |\ x_{1}{'}, x_{2}{`}, \dots, x_{m}{`}, y_{1}, y_{2}, \dots, y_{n}) = P(x_{1}, x_{2}, \dots, x_{l} | y_{1}, y_{2}, \dots, y_{n}) } \]
\end{itemize}

\section{Training Bayesian Classifier}\label{sec:training-bayesian-classifier}
During \emph{training}, typically \emph{log-space} is used.

\data{\begin{equation*}
		  \begin{aligned}
			  y_{NB} &= \arg\max_{y} \left[ \log P(y) \Pi_{i=1}^{n} P(x_{i} | y) \right]\\
			  &= \arg\max_{y} \left[ \log P(y) + \sum_{i=1}^{n} \log P(x_{i} | y) \right]\\
		  \end{aligned}
\end{equation*}}

\section{Text Classification}\label{sec:text-classification}
\begin{algorithm}[H]
	\caption{Text-based Naïve Bayes Classification}\label{alg:train-naive-bayes}
	\begin{algorithmic}[1]
		\Function{Train-Naive-Bayes}{$D, C$} \Returns $\log P(c)$ and $\log P(w|c)$
		\ForAll{class $c\in C$}\Comment{Calculate $P(c)$ terms}
		\State $N_{doc} \gets$ number of documents in $D$
		\State $N_{c} \gets$ number of documents from $D$ in class $c$
		\State $logprior[c] \gets \log\frac{N_{c}}{N_{doc}}$
		\State $V \gets $ vocabulary of $D$
		\State $bigdoc[c] \gets \Call{Append}{d}$ \textbf{for} $d\in D$ \textbf{with} class $c$
		\ForAll{word $w$ in $V$}\Comment{Calculate $P(w|c)$ terms}
		\State $\Call{Count}{w,c}\dots$
		\EndFor
		\EndFor
		\EndFunction
	\end{algorithmic}
\end{algorithm}

The word \emph{with} doesn't occur in the training set, so we drop it completely (we don't use unknown word models for Naïve Bayes)

\section{Evaluating Classifiers}\label{sec:evaluating-classifiers}
\begin{itemize}
	\item \emph{Gold Label} is the \emph{correct} output \emph{class} label of an input.
	\item \emph{Confusion Matrix} is a table for \emph{visualizing} how a \emph{classifier performs} with respect to the fold labels, using two dimensions (system output and gold labels), and each cell labeling a set of possible outcomes.
	\item \emph{True Positives} and \emph{True Negatives} are \emph{correctly classified} outputs belonging to the positive and negative class, respectively.
\end{itemize}

\section{Precision, Recall, F-Measure}\label{sec:precision-recall-f-measure}
\begin{equation}
	\mathbf{Precision} = \frac{\mbox{true positives}}{\mbox{true positives } + \mbox{ false positives}}
	\label{eq:precision}
\end{equation}

\section{ROC Curve}\label{sec:roc-curve}
\begin{itemize}
	\item A receiver operating characteristic curve (ROC curve) is a graphical plot that illustrates the \emph{performance} of a \emph{binary classifier model}.
	\item The \emph{ROC curve} is the plot of the \emph{true positive rate (recall)} (TPR) against the \emph{false positive rate} (FPR).
	\item \emph{ROC curve} plots \emph{TPR vs. FPR} at different \emph{classification thresholds}.
	\item \emph{Classification threshold} is used to convert the \emph{output} of a \emph{probabilistic classifier} into class \emph{labels}.
	\item The \emph{threshold} determines the
\end{itemize}

\section{\naive Bayes: Two Classes}\label{sec:naive-bayes:-two-classes}
\begin{itemize}
	\item \naive Bayes classifier gives a method for \emph{predicting} the \emph{most likely class} rather than an explicit class.
	\item In the case of two classes, \data{$y\in\{0, 1\}$} we predict that \data{$y=1$} iff
	\[ \dots \]
\end{itemize}

Take logarithm;

\[ \log \frac{P(y_{j}=1)}{P(y_{j}=0)} + \sum_{i}\log \frac{1-p_{i}}{1-q_{i}} + \sum_{i}\left( \log\frac{p_{i}}{1-p_{i}} - \log\frac{q_{i}}{1-q_{i}} \right)x_{i} > 0 \]
\begin{itemize}
	\item We get that \naive bayes is a \emph{linear separator} with --
	\[ w_{i} = \log\frac{p_{i}}{1 - p_{i}} - \log\frac{q_{i}}{1 - q_{i}} = \log\frac{p_{i}(1-q_{i})}{q_{i}(1-p_{i})} \]
	\item In the case of two classes, we can say:
	\item but since \data{$P(y_{j} = 1|x) = 1 - P(y_{j} = 0 | x)$}, we get:
	\[ P(y_{j}=1|x) = \frac{1}{1+e^{-\left( \sum_{i}w_{i}x_{i} + b \right)}} \]
	\item This is logistic regression
\end{itemize}

\end{document}
